%%%%%%%%%%%%%%%%%%%%%%%%%%%%%%%%%%%%%%%%%
% University/School Laboratory Report
% LaTeX Template
% Version 3.1 (25/3/14)
%
% This template has been downloaded from:
% http://www.LaTeXTemplates.com
%
% Original author:
% Linux and Unix Users Group at Virginia Tech Wiki 
% (https://vtluug.org/wiki/Example_LaTeX_chem_lab_report)
%
% License:
% CC BY-NC-SA 3.0 (http://creativecommons.org/licenses/by-nc-sa/3.0/)
%
%%%%%%%%%%%%%%%%%%%%%%%%%%%%%%%%%%%%%%%%%

%----------------------------------------------------------------------------------------
%	PACKAGES AND DOCUMENT CONFIGURATIONS
%----------------------------------------------------------------------------------------

\documentclass{article}
\usepackage{graphicx}
\usepackage{subcaption}
\usepackage{wrapfig}
\graphicspath{ {.} }
\usepackage{listings}
\usepackage{multirow}
\renewcommand\thesubsection{\arabic{subsection}}
\renewcommand\thesubsubsection{\alph{subsubsection}}
\usepackage[justification=centering]{caption}
\usepackage{booktabs}
\usepackage{hyperref}
\hypersetup{
    colorlinks=true,
    linkcolor=blue,
    filecolor=magenta,      
    urlcolor=blue,
}
\usepackage[margin=1.5in]{geometry}
\usepackage{float}
\usepackage{array}
\usepackage{adjustbox}
\newcolumntype{C}[1]{>{\centering\let\newline\\\arraybackslash\hspace{0pt}}m{#1}}


%----------------------------------------------------------------------------------------
%	DOCUMENT INFORMATION
%----------------------------------------------------------------------------------------

\title{CS 6241: Compiler Design \\~\\ Solving Constant Propagation Data-flow Problem Using Detected Destructive Merges Information} % Title

\author{Chayne \textsc{Thrash} and Mansour \textsc{Alharthi}} % Author name

\date{\today} % Date for the report

\begin{document}

\maketitle % Insert the title, author and date

\begin{center}
\begin{tabular}{l r}
Instructor: & Professor Santosh \textsc{Pande} % Instructor/supervisor
\end{tabular}
\end{center}

\subsection{Introduction}
In this project, we implemented the algorithm introduced in paper "Comprehensive Path-sensitive Data-flow Analysis" by Thakur and Govindarajan. The algorithm first detects the nodes where two or more paths merge; causing constant variables defined in one or more of those paths no longer feasible for propagating. After that, the algorithm look for the influenced nodes; the nodes of which will be optimized in case the destructive merge is eliminated. Finally, the algorithm duplicates the nodes that are in the \textit{region of influence}, and thus eliminating the destructive merge allowing for the constant propagation to the influenced nodes. The algorithm represents a trade-off between code size and data-flow precision. We apply the algorithm on only the two top fittest destructive merges, as per the definition of fitness in the paper. 

\subsection{Test Results}
We tested our implementation on two benchmarks. We compare our implementation with sparse conditional constant propagation technique by Wegman-Zadeck implemented in LLVM. 

~\\~

\begin{adjustbox}{center}
\renewcommand{\arraystretch}{2}
%\begin{table}[]
\begin{tabular}{| C{3cm} | C{3cm} | C{3cm} | C{3cm} | C{3cm} |}
\hline
Benchmark & Static size with sparse conditional constant propagation & Static size with destructive merge elimination & \% of increase in static size \\ \cline{1-4} 
H.264/MPEG-4 AVC & A & B & C  \\ \cline{1-4}
JPEG-2000 & D & E & F  \\ \cline{1-4}
\end{tabular}
%\end{table}
\end{adjustbox}
\captionof{table}{Static size comparison} 


\begin{adjustbox}{center}
\renewcommand{\arraystretch}{2}
%\begin{table}[]
\begin{tabular}{| C{3cm} | C{3cm} | C{3cm} | C{3cm} | C{3cm} |}
\hline
Benchmark & Constants propagated\# with sparse conditional constant propagation & Constants propagated\# with destructive merge elimination & \% of Constants propagated increased \\ \cline{1-4} 
H.264/MPEG-4 AVC & A & B & C  \\ \cline{1-4}
JPEG-2000 & D & E & F  \\ \cline{1-4}

\end{tabular}
%\end{table}
\end{adjustbox}
\captionof{table}{Constants propagated comparison} 

~\\~
\begin{adjustbox}{center}
\renewcommand{\arraystretch}{2}
%\begin{table}[]
\begin{tabular}{| C{3cm} | C{3cm} | C{3cm} | C{3cm} | C{3cm} |}
\hline
Benchmark & Execution time with sparse conditional constant propagation & Execution time with destructive merge elimination & \% of improvement in Execution time \\ \cline{1-4} 
H.264/MPEG-4 AVC & A & B & C  \\ \cline{1-4}
JPEG-2000 & D & E & F  \\ \cline{1-4}
\end{tabular}
%\end{table}
\end{adjustbox}
\captionof{table}{Execution time comparison} 
\subsection{Work Breakdown}
~\\~
\begin{adjustbox}{center}
\renewcommand{\arraystretch}{2}
%\begin{table}[]
\begin{tabular}{| C{3cm} | C{3cm} | C{3cm} | C{3cm} | C{3cm} |}
\hline
Destructive Merges Detection & Split Graph and CFG Reconstruction & Preliminary Reachability Analysis & Report & Integrating and testing \\ \cline{1-5} 
Mansour & Chayne & Chayne & Mansour &  \\ \cline{1-5}

\end{tabular}
%\end{table}
\end{adjustbox}
\captionof{table}{breakdown of work among team members} 

\end{document}